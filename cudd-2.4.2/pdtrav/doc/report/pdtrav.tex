\documentstyle[11pt]{article}

\setlength{\textheight}{8.75in}
\setlength{\textwidth}{7.0in}
\setlength{\footheight}{0.0in}
\setlength{\topmargin}{0.25in}
\setlength{\headheight}{0.0in}
\setlength{\headsep}{0.0in}
\setlength{\oddsidemargin}{-.35in}
\setlength{\parskip}{1ex plus .2ex minus .2ex}
\setlength{\parindent}{0pc}
\setcounter{topnumber}{5}
\setcounter{dbltopnumber}{5}

\newcommand{\GpC}[1]{{\bf GpC: #1 \\}}
\newcommand{\StQ}[1]{{\bf StQ: #1 \\}}

\title{\Large\bf
PdTRAV \\
Politecnico di Torino Reachability Analysis for Verification \\
(Release 2.0Beta)
}

\author{
\parbox{1.5in}{\begin{center}
               Gianpiero Cabodi
               \end{center}}
\parbox{1.5in}{\begin{center}
               Stefano Quer
               \end{center}}
\\~\\
\parbox{3in}{\begin{center}
             Politecnico di Torino \\
             Dip.\ di Automatica e Informatica \\
             Turin, ITALY
             \end{center}}
}

%\date{~}

\begin{document}

\maketitle
\thispagestyle{empty}
\small

%###############################################################################

\newpage

\setcounter{page}{1}
\normalsize
\tableofcontents

\newpage

%###############################################################################
%###############################################################################
%###############################################################################

\section{Introduction} 

The Politecnico di Torino Reachability Analysis for Verification
(PdTRAV) package provides functions to perform BDD based
symbolic reachability analysis of Finite State Machines (FSMs).
The functions are intended for use within the field of formal verification
of sequential circuits. 

The package can be used as:
\begin{itemize}
\item
as a black box, i.e., a library of functions to be linked with an application
\item
as a clean box which may be worked into and expanded by a programmer.
\end{itemize}

The CMD sub--package provides a simple textual user interface for
users who simply want to perform reachability and BDD operations.

The package includes the following sub--packages (directories):
\begin{itemize}
\item
Decision Diagram Interface (DDI): almost all BDD package dependencies are
concentrated here.
DDI can also be viewed as a portability interface for a chosen BDD package.
Moreover, it provides some higher level abstractions such as BDD based
types (monolithic/partitioned Boolean functions, arrays of Boolean functions,
variables, variable arrays and sets).
The present implementation relies on the CUDD package by Fabio Somenzi
(version 2.3.0).
\item
Utility (PDTUTIL): memory management, string manipulation and
other low level non BDD routines. 
\item
CUDD interface (CUPLUS): this directory contains some 
extensions to the CUDD package.
All BDD package extensions (mostly recursive functions working of BDD nodes)
have been (and should be) created in this directory, or in a similar one in
the case of porting to another BDD package. 
\item
Finite State Machine (FSM): this package contains routines for storing/loading
and manipulating a FSM structure.
The BLIF format is presently supported for FSM netlists. We also
introduce a FSM dump format based on the DDDMP BDD format 
(see dddmp package inside the CUDD package).
\item
Transition Relation (TR): symbolic traversals are based on transition
relations.
This directory provides manipulation functions, i.e., sorting, partitioning,
clustering, for monolithic and partitioned transition relations.
Support for transitive closure is also included.
\item
Traversal (TRAV): image computations and full traversals are available from
this package. Invariant (assertion based) checking, is included.
\item
Conjunctive and disjunctive partitioning (PART).
This directory supplies the user with routines to conjunctively and
disjunctively partition a BDD.
\item 
Command user interface (CMD): a textual interactive user interface for the
package.
It allows to issue commands and to parse them by calling the functionalities
of the other packages.
\end{itemize}

Mutual interactions and references among sub--packages exist internally.
Some of the sub--packages have been build up mainly as bottom layers for other sub--packages,
but they could be used as stand--alone packages as well by any application.
E.g. the DDI package is the only one required for an application which
requires Boolean function manipulation only, without requiring the
FSM, transition relation and traversal packages.
In the following we assume that the reader is familiar with BDDs, basic
operations using BDDs, and symbolic traversals.
The reader should refer to~\cite{Bryant86,Bryant92} for a survey on these
topics.

%###############################################################################
%###############################################################################
%###############################################################################

\section{How to Get and Install PdTRAV} 

%###############################################################################

\subsection{The PdTRAV Package} 

The PdTRAV package is available via anonymous FTP from ...
A compressed tar file named pdtrav-2.0.tar.gz can be found in directory 
pub.
Once you have this file, 
\begin{verbatim}
gzip -dc pdtrav-2.0.tar.gz | tar xvf - 
\end{verbatim}
will create directory pdtrav-2.0 and its subdirectories.
These directories contain the package, including documentation and a few usage
examples.

There is a README file with instructions on configuration and installation
in pdtrav-2.0. 
You can use a compiler for either ANSI C or C++. 
To compile and link the package the user should run the main {\sf Makefile}
in the src directory (generated by autoconf).
The user can choose different setting options depending on
the hardware/OS on which the installation has to be performed.
The {\sf Cudd-2.3.0} package is required by pdtrav
on a sibling directory of the PdTrav.
Information on the {\sf Cudd-2.3.0} can be found from
{\sf http://vlsi.colorado.edu/~fabio}.

Once the user has made the libraries and executable program, he can test
the program in the {\sf exp} directory.
Such a directory contains a few {\sf blif} files on which to run
PdTRAV (cmd interface), and
a few script files showing how to run it.
From the root directory type, for example:
\begin{verbatim}
cd <root directory>/pdtrav-2.0/exp/trav
../../bin/pdtrav
source s1423TravOrdTrMon.cmd
\end{verbatim}
This runs the traversal program on the {\sf s1423} circuit and with a
variable ordering optimized for the monolithic transition relation
representation.
On a $400$MHz Pentium II processor with $128$ Mbytes of main memory and
running Linux RedHat 5.2 Apollo it takes about $40$ seconds.
The output produced by the program can be checked against the contents of
the file s1423TravOrdTrMon.out.
More information on the pdtrav program can be found in the README file on
the root of the distribution. 
(ImgTime: 0.65 sec)(TotalTime: 3.19 sec)
TravLevel 4: [|Tr|: 758881] [|From|: 556][|To|: 1228][|Reached|: 1397][\#ReachedStates: 392225]
(ImgTime: 1.19 sec)(TotalTime: 4.89 sec)
TravLevel 5: [|Tr|: 758881] [|From|: 1275][|To|: 3168][|Reached|: 3316][\#ReachedStates: 2.08012e+06]
(ImgTime: 2.71 sec)(TotalTime: 8.34 sec)
TravLevel 6: [|Tr|: 758881] [|From|: 3152][|To|: 7496][|Reached|: 7763][\#ReachedStates: 8.49328e+06]

Another possibility is the {\sf src/test} directory, which shows some
examples of how to use pdtrav as a library. The {\sf test} executable
(testPdtrav) is moved in the bin directory, too.

If you want to be notified of new releases of the PdTRAV package, send a 
message to the authors {cabodi,quer}@polito.it.
Moreover, send feedback to the same authors.
 
%###############################################################################
%###############################################################################
%###############################################################################

\section{User's Manual} 

This section describes the use of the PdTRAV package and sub--packages as a
black box. 

%###############################################################################

\subsection{Compiling and Linking}
 
To build an application that uses PdTRAV, you should add 
a line 

\begin{verbatim}
#include "<pkg>.h" 
\end{verbatim}

to your source files, and should link $lib<pkg>.a$ (for any given
$<pkg>$ within PdTRAV you need) to your executable. 
Some platforms require specific compiler and linker flags.
Refer to the Makefile in the top level directory of the distribution. 

A brief description of the main sub--packages packages follows.
They are characterized by a set of functions and manager structure
(accessed to by pointer) describing a consistent working
environment for operations.
Moreover, DDI defines the BDD based data types used by our package.

%###############################################################################

\subsection{The DDI package}

This package is intended as a portability interface and a higher level 
abstraction of the BDD package. It is used by all other PdTRAV packages
working with BDDs, but it might be used as a stand--alone package as well.

A {\bf DDI manager} is defined in quite a similar way CUDD does. In summary,
it is a structure (type Ddi\_Mgr\_t)
containing all informations about a given BDD environment,
with a set of variables, hash/cache tables, and BDD nodes/functions.
DDI uses its own descriptor (a "C" structure), which, besides pointing to a CUDD manager,
allows handling extra informations, like variable names and auxiliary indexes.
These are additional variable attributes (names and integer indexes), which
are particularly useful as absolute variable references.
We introduced them for use in multiple manager frameworks or whenever an
application involves several runs of the same program, without a constant
correspondence between variable indexes and physical entities
(circuit inputs/outputs/latches).


%###############################################################################

\subsubsection{Basic Data Structures} 

Besides extending manager capabilities, DDI introduces strong typing
for objects represented using BDDs, and uses ``handles'' as a
mechanism for allocation/freeing. More specifically, four types 
are presently defined and managed as abstract data types, 
all available in scalar and array form:

\begin{itemize}

\item
{\bf Variable (Ddi\_Var\_t, Ddi\_Vararray\_t)}
A variable is uniquely referenced by an integer index within a given manager, 
and optionally by a variable name and/or an auxiliary index (auxid). 
A variable $v$ is also caracterized by a position in variable
ordering, which is not fixed (and generally differs from the index) if
dynamic variable ordering is enabled. 
A variable is often represented in BDD packages (e.g. CUDD) by the
corresponding positive literal (a BDD).
We clearly distinguish a variable $v$ from the corresponding
literals ($v$ and $\overline{v}$), which
are in DDI Boolean functions (type {\bf Ddi\_Bdd\_t}).
Variable arrays are useful whenever we need groups of variables with
well
defined positions, for instance to establish a correspondence between
present and next state variables (in FSM traversal), or between
Boolean functions and variables (in compose operations).

\item
{\bf Variable Sets (Ddi\_Varset\_t, Ddi\_Varsetarray\_t)}
A variable set is a group of variables with no defined position or
ordering. Variable sets are represented in some packages (e.g. SIS) by
bit vectors, in CUDD by BDD cubes.
DDI internally uses the latter strategy, and externally provides 
usual set operators. Variable sets are useful as inputs for
support management and quantification operators.

\item
{\bf Boolean Functions (Ddi\_Bdd\_t, Ddi\_Bddarray\_t)}
A Boolean function is either a monolithic, partitioned (conjunctively
or disjunctively) or meta BDD. We adopt the BDD 
acronym to emphasize the Decision Diagram based representation.
Partitioned representations are recursive. The terms of a
conjunctive form are BDDs, so they may be partitioned themselves.
Arrays of BDDs are useful whenever we need to represent vectors of
functions, e.g. for output or next state functions of circuits. 
Internally, partitioned BDDs are stored as trees with monolithic BDDs
as leaves.
Meta BDDs are a particular BDD representation described in
\cite{MetaBDDs}.

\item
{\bf Expressions (Ddi\_Expr\_t, Ddi\_Exprarray\_t)}
Expressions are more general than BDDs. They may include BDDs (since
monolitic and partitioned BDDa are Boolean expressions), but they are
expecially concieved for invariants and temporal logic (e.g. CTL)
formulas.
Moreover, expression leaves are not necessarily BDDs. They might be
string identifiers, to be associated to BDDs at later steps of symbolic 
manipulations.
 
\end{itemize}

All of the above types are implemented by resorting to handles. The
handle mechanism hides the ref/deref BDD node allocation style
implemented in CUDD. Handles are dynamically allocated structures 
pointing to CUDD BDD nodes, and only accessed to by pointers.
Handles are allocated, freed, duplicated. A handle represents a
reference to a BDD node, so a ref/reref operation is implicitly done
any time a handle is created/released. The only exception to this
general rule are variables, which are never duplicated nor freed (except
when freeing a DDI manager)

Handles are expecially useful as an intermediate portability layer
between the
higher levels and CUDD\footnote{Other BDD packages could be introduced
by partly rewriting DDI}. Moreover they enable storing additional
informations characterizing DDI entities, like name, code, extra
pointers.
Names are expecially concieved for debug and/or log purposes, codes
can distinguish particular subtypes (e.g. monolithic/partitioned forms
or operators in expressions). We use pointers for list based
management of allocated DDI objects, and to keep a reference to the
owner DDI manager (this avoids passing manager to most functions).

We call the above data types {\bf DDI nodes}.
%%GENERIC TYPE

%##############################################################################

\subsubsection{Operators}

An operation may either generate a new result DDI node, or modify an
existing one, following the accumulator operation style. 
Most operations returning scalar data are provided with
both options. They gerate a new DDI node (e.g. c = Ddi\_BddAnd(a,b)), 
or they accumulate (e.g. Ddi\_BddAndAcc(a,b))
the result in the first operand, after freeing old data). Conversely,
operations on array data always modify an existing DDI node (explicit
calls to proper ``Alloc'' or ``Dup'' functions are required to generate
new arrays). Going to operand data, whenever an operation requires 
data transfer to a destination DDI node, two options are available:
{\em copying} or {\em moving}. In the former case, data are duplicated
(new DDI nodes are generated and CUDD nodes implicitly referenced), in
the latter one pointers (partitions, sub-expressions, new array
entries) are copied and actually moved to a new owner DDI node.

{\sf Make} is used in names to enforce creation (allocation) of new
data (e.g. {\sf Ddi\_BddMakeMono} creates a monolitic BDD from a non
monolitic one), {\sf Set } is used to specify data transformation
directly one the parameter (e.g. {\sf Ddi\_BddSetMono} transforms a non 
monolitic BDD to monolitic). 

Manipulations on non scalar objects
(e.g. arrays and partitioned BDDs) support both {\em insert/extract}
and {\em read/write} operations. {\sf Insert} always creates a new
array slot (or partition), {\sf Extract} removes the returned slot
(partition), {\sf Write} overwrites (and frees) an existing array slot
(partition). {\sf Insert} and {\sf Write} always generate a duplicate
of the manipulated object (so that the received parameter is still
owned by the calling procedure), {\sf Extract} and {\sf Read} make no 
duplicate of the returned object (but the result of {\sf Extract} is
owned by the calling procedure after removal from array or partitioned BDD).

The following operators are presently supported:

\begin{itemize}

\item 

DDI-CUDD conversions. Functions performing conversions between
different types are named $Ddi\_<type>MakeFromCU$ and $Ddi\_<type>ToCU$ 
(being type one of the Ddi types) are provided for conversions from/to CUDD 
objects. Conversions from CUDD always generate a new DDI node (with
implicit node referencing), whereas
their dual ...ToCU counterparts simply read CUDD node pointers (with
no node referencing). Memory allocation is done only in case of
arrays (e.g. Ddi\_BddarrayToCU), which require a dynamically allocated
array of pointers.

\item

Management of (conjunctively and disjunctively) partitioned forms. 

\item
Standard BDD operators: ITE, AND, OR, ... AND-EXIST, ..., constrain, 
restrict ...
They are implemented by resorting to the corresponding CUDD functions.
Most operators support partitioned BDDs, with restrictions as
described in the DDI package documentation.

\item 
Load/store to file. Boolean functions and variables may be stored to file.
The functions are implemented through the {\em dddmp} package
(distributed with CUDD) which provides efficient dumping of BDDs and arrays of BDDs.
Partitioned forms are managed by proper headers (considered as comments by dddmp)

\end{itemize}

It must be noted that the results of all operators are implicitly referenced
(unless explicitly stated in the function documentation), while explicit 
free-ing is required for all types.

%###############################################################################


\subsection{FSM}

Each FSM has an internal representation.
Each internal representation can be stored on secondary memory on what we
call a FSM file.
This gives the user the possibility to store intermediate computation
results and to exchange information on a FSM and BDD basis.
A result of this strategy is to allow the user to perform with great
flexibility a certain number of (expensive) operations such as partially
traverse a FSM, store intermediate results, and restart reachability analysis
afterwards.

Internal and external (on files) formats are pretty alike one to the other.
They contain information about the size of the FSM (number of inputs,
outputs, latches), names, and BDD representations (for initial state set,
next states and output functions, transition relations, and so on).
The current format is the following one:

{\small
\begin{verbatim}
.Fsm <name>

.Size
  .i <Number of Primary Inputs>
  .o <Number of Primary Outputs>
  .l <Number of Memory Elements>
.EndSize

.Ord
  .ordFile <File Name for the Variable Order File>
.EndOrd

.Name
  .i <Name of Primary Input Variables>
  .ps <Name of Present State Variables>
  .ns <Name of Next State Variables>
.EndName

.Index
  .i <Variable Index for Primary Inputs>
  .ps <Variable Index for Present State Variables>
  .ns <Variable Index for Next State Variables>
.EndIndex

.Delta
  .bddFile <File Name for the Next State Functions BDD File>
.EndDelta

.Lambda
  .bddfile <File Name for the Output Functions BDD File>
.EndLambda

.InitState
  .bddFile <File Name for the Initial State Set BDD File>
.EndInitState

.Tr
  .bddFile <File Name for the Transition Relation BDD File>
.EndTr

.Reached
  .bddFile <File Name for the Reachable State Set BDD File>
.EndReached

.EndFsm
\end{verbatim}
}

In general, not all the sections are present, depending on the ``state''
of the program.
For example, the reachable state set at the beginning does not exist and
storing the FSM implies not storing the section
{\small
\begin{verbatim}
.Reached
  .bddfile <File Name for the Reachable State Set BDD File>
.EndReached
\end{verbatim}
}

Functions are given to perform a certain number of operations on the
FSM structure.
Namely we allow the user to:
\begin{itemize}
\item
Initialize an internal representation for the FSM
\item
Duplicate an internal representation for the FSM
\item
Destroy an internal representation for the FSM
\item
Store a FSM representation with the previous format
\item
Load a FSM representation.
\end{itemize}
Given the name of the FSM the names of the various files are automatically
created and dealt with.
Typically if {\sf name} is the name of the FSM file, {\sf name.ord} is the
name of the variable order file, {\sf namedelta.bdd} is the name of the file
containing the next state function BDD representation, and so on.
During storing and loading operations two modes are permitted: In the first
mode the FSM file and all the BDD files are loaded, in the second mode only the
FSM file is loaded.
This features allows the user to explicitly concentrate on the objects
he really cares about.
For example next state functions, the transition relation and the reachable
state set can be loaded separately and optimized with different variable
orders in different working sections.
The package provides functions to manage (such as load and store or perform
operation on) singular object belonging to a FSM (such as BDDs).
Normally the FSM description has to be loaded or created before any other
object belonging to it (such as the transition relation).

%###############################################################################

\subsection{TR management}

Functions for creating and manipulating transition relations are provided by
this package.
More specifically:

\begin{itemize}
\item TR creation: from next--state functions and variable arrays
(inputs, present/next--state).
From scratch (GpC{To be done.}).
\item TR sorting: variants of the IWLS95 heuristics are provided, the goal
being a good partition sorting for efficient early smoothing with conjunctive 
partitioning
\item TR clustering: linear conjunction of partitions controlled by a 
size threshold.
\item TR subsetting: an external constraint is used to constrain a TR
by means of conjunction.
\item TR partitioning
\item Abstractions oriented to approximate reachability.
\end{itemize}

%###############################################################################

\subsection{Traversals}

This package provides several routines for image/preimage computations, 
traversals,
transitive closure of a State Transition Graph. All tasks are accomplished 
working on transition relations and characteristic functions of state sets.

A traversal manager includes all data involved in a traversal, namely the TR,
state sets, settings, etc.
Standard traversals as well as more specialized ones are supported.

%###############################################################################

\subsection{The CMD user interface}

PdTRAV is equipped with an interactive textual user interface.
It is intended for users who plan not to link PdTRAV packages as an object 
library.
Commands are issued on a line by line basis.

If the first character of a command is the exclamation mark "!", a system
call is issued instead.
This allows to execute basic system commands such as "more", "ls", etc.

If the first character of a command is "\#" the command is ignored, i.e., it
is supposed to be a comment.
This is especially useful to write script files.
In fact, the command "source" gives the possibility to read from a
source-script file instead that from standard input.

If the first character of the command is "@" this indicates that a script
file with parameter is used.
For example
{\small
\begin{verbatim}
@script1.cmd a b c
\end{verbatim}
}
runs the script file script1.cmd (containing some \$1, \$2 and \$3) with
parameters a, b and c.

For more information on the commands see the documentation in the {\sf cmd\\doc}
directory and the on-line documentation (command {\sf help}).

%###############################################################################

\subsection{Setting Parameters}

All packages require parameter setting for their operations.
For each package, proper functions support set/show and restoring of default
settings package provides several functions to set the parameters that control
various functions.
The value of these settings are usually stored into the relative manager
structure.
A lot of commands can receive additional options.
These options usually overwrite the generic settings.

%###############################################################################
%###############################################################################
%###############################################################################

%\section{An example}
%

%###############################################################################
%###############################################################################
%###############################################################################

\section{Programmer's Manual} 

This section should provide additional detail on the working of the PdTRAV
package and on the programming conventions followed in its writing.
The additional detail should help those who want to write procedures that
directly manipulate the DDI and other PdTRAV data structures. 
Anyway, the Programmer manual, for a visibility of the package as a clear
box, {\bf is not provided in this release.} 

%###############################################################################
%###############################################################################
%###############################################################################

\section{Acknowledgments} 

Contributors and early adopters to be defined.

%###############################################################################
%###############################################################################
%###############################################################################

\bibliographystyle{unsrt}
%\bibliographystyle{plain}
%\bibliography{/home/cabodi/pap/dac99/biblio}
\bibliography{/disk1/paper/biblio/biblio}

\end{document}
